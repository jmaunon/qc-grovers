\begin{frame}
\frametitle{Inicialización}
El primer paso es el de \textit{inicializar} nuestro sistema. Para ello se hace uso del principio de superposicíon, es decir, en lugar de buscar en un único lugar, buscamos en varios al mismo tiempo.
\pause
\begin{eqnarray}
(I^{\otimes n}\otimes X)\ket{0}_{n+1}&=&\ket{0}_n\otimes\ket{1}
\end{eqnarray}
\end{frame}

%\begin{frame}
%\frametitle{Definiciones básicas}
%
%\begin{block}[Producto tensorial]
%  Dado $A\in\mathbb{C}^{m\times n}$, $B\in\mathbb{C}^{p\times q}$, el producto tensorial $A\otimes B$ es la matriz            $D\in\mathbb{C}^{pm\times nq}$ tal que:
%	\[
%		D:=A\otimes B = 		
%		\begin{pmatrix}
%    a_{11}B & \cdots & a_{1n}B \\
%    a_{21}B & \cdots & a_{2n}B \\
%    \vdots  &        &  \vdots \\
%    a_{m1}B & \cdots & a_{mn}B
%  		\end{pmatrix}
%	\]
%\end{block}
%
%
%\begin{block}
%  \[
%  x\otimes y = \colvec{2}{1}{0}\otimes\colvec{2}{0}{1}=\colvec{4}{0}{1}{0}{0}
%  \]
%\end{block}
%
%\end{frame}
