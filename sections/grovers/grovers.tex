\begin{frame}
\frametitle{Inicialización}
El primer paso es el de \textit{inicializar} nuestro sistema. Para ello se hace uso del principio de superposicíon, es decir, en lugar de buscar en un único lugar, buscamos en varios al mismo tiempo.
\pause
\begin{eqnarray}
(I^{\otimes n}\otimes X)\left.|0\right\rangle _{n+1}&=&\left.|0\right\rangle _{n}\otimes\left.|1\right\rangle \nonumber\\
H^{\otimes\left(n+1\right)}\left[(I^{\otimes n}\otimes X)\left.|0\right\rangle _{n+1}\right]&=&H^{\otimes n}\left.|0\right\rangle _{n}\otimes H\left.|1\right\rangle \nonumber\\
&=&\sum_{j\in\{0,1\}^{n}}\frac{1}{\sqrt{2^{n}}}\left.|j\right\rangle _{n}\otimes\frac{1}{\sqrt{2}}\left(\left.|0\right\rangle -\left.|1\right\rangle \right)\nonumber\\
&=&\sum_{j\in\{0,1\}^{n}}\alpha_{j}\left.|j\right\rangle _{n}\otimes\frac{1}{\sqrt{2}}\left(\left.|0\right\rangle -\left.|1\right\rangle \right)\nonumber\\
&=&\psi_{n}\times\psi_{1}\nonumber
\end{eqnarray}
\end{frame}

\begin{frame}
\frametitle{Definiciones básicas}

\begin{exampleblock}{Inicialización}
\begin{eqnarray}
(I^{\otimes3}\otimes X)\left.|0\right\rangle _{3+1}&=&I^{\otimes3}\left.|0\right\rangle _{3}\otimes X\left.|0\right\rangle \\&=&\left(\begin{array}{cccccccc}
1\\
 & 1\\
 &  & 1\\
 &  &  & 1\\
 &  &  &  & 1\\
 &  &  &  &  & 1\\
 &  &  &  &  &  & 1\\
 &  &  &  &  &  &  & 1
\end{array}\right)\left(\begin{array}{c}
1\\
0\\
0\\
0\\
0\\
0\\
0\\
0
\end{array}\right)\otimes\left(\begin{array}{cc}
0 & 1\\
1 & 0
\end{array}\right)\left(\begin{array}{c}
1\\
0
\end{array}\right)\\&=&\left.|0\right\rangle \otimes\left.|1\right\rangle 
\end{eqnarray}
\end{exampleblock}


\end{frame}

\begin{frame}
\begin{exampleblock}{}
\begin{eqnarray}
&&H^{\otimes\left(3+1\right)}\left[(I^{\otimes3}\otimes X)\left.|0\right\rangle _{n+1}\right]\\&=&H^{\otimes3}\left.|0\right\rangle \otimes H\left.|1\right\rangle \\&=&\frac{1}{\sqrt{2^{3}}}\left(\begin{array}{cc}
1 & 1\\
1 & -1
\end{array}\right)\left(\begin{array}{c}
1\\
0
\end{array}\right)\otimes\frac{1}{\sqrt{2}}\left(\begin{array}{cc}
1 & 1\\
1 & -1
\end{array}\right)\left(\begin{array}{c}
0\\
1
\end{array}\right)\\&=&\frac{1}{\sqrt{2^{3}}}\left(\begin{array}{cccccccc}
1 & 1 & 1 & 1 & 1 & 1 & 1 & 1\\
1 & -1 & 1 & -1 & 1 & -1 & 1 & -1\\
1 & 1 & -1 & -1 & 1 & 1 & -1 & -1\\
1 & -1 & -1 & 1 & 1 & -1 & -1 & 1\\
1 & 1 & 1 & 1 & -1 & -1 & -1 & -1\\
1 & -1 & 1 & -1 & -1 & 1 & -1 & 1\\
1 & 1 & -1 & -1 & -1 & -1 & 1 & 1\\
1 & -1 & -1 & 1 & -1 & 1 & 1 & -1
\end{array}\right)\left(\begin{array}{c}
1\\
0\\
0\\
0\\
0\\
0\\
0\\
0
\end{array}\right)\otimes\frac{1}{\sqrt{2}}\left(\begin{array}{cc}
1 & 1\\
1 & -1
\end{array}\right)\left(\begin{array}{c}
0\\
1
\end{array}\right)\\&=&\frac{1}{\sqrt{2^{3}}}\left(\begin{array}{c}
1\\
1\\
1\\
1\\
1\\
1\\
1\\
1
\end{array}\right)\otimes\frac{1}{\sqrt{2}}\left(\begin{array}{c}
1\\
-1
\end{array}\right)\\&=&\frac{1}{2\sqrt{2}}\left.|000\right\rangle +\frac{1}{2\sqrt{2}}\left.|001\right\rangle +\frac{1}{2\sqrt{2}}\left.|010\right\rangle +\frac{1}{2\sqrt{2}}\left.|011\right\rangle \\&+&\frac{1}{2\sqrt{2}}\left.|100\right\rangle +\frac{1}{2\sqrt{2}}\left.|101\right\rangle +\frac{1}{2\sqrt{2}}\left.|110\right\rangle +\frac{1}{2\sqrt{2}}\left.|111\right\rangle \\&\otimes&\frac{1}{\sqrt{2}}\left(\left.|0\right\rangle -\left.|1\right\rangle \right)
\end{eqnarray}
\end{exampleblock}
\end{frame}

\begin{frame}
\empty
\end{frame}
