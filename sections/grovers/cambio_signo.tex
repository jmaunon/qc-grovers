\begin{frame}
	\frametitle{Sign flip}
	We find a method (unitary operator) which flip the sign of the state of interest.
	OJO: AÑADIR GRAFICO QUE pase de inicialización a cambio de signo gracias a $U_f$ 
\end{frame}
\begin{frame}
	\frametitle{Sign flip}
	\begin{block}{\textit{Quantum Oracle}}
		It is defined the operator $U_f$: 
		\[U_{f}:\left.|j\right\rangle _{n}\otimes\left.|y\right\rangle _{1}\rightarrow\left.|j\right\rangle _{n}\otimes\left.|y\oplus f\left(j\right)\right\rangle _{1},\]
		where $\oplus$ is the sum operator in mod 2, and $f\left(j\right)=\left\{ \begin{array}{cc}
		1  & j=l\\
		0  & j\neq l
		\end{array}\right\} $
	\end{block}
	
	
	\begin{center}
		\begin{tabular}{|c c |c|} 
			\hline
			A & B & XOR \\ 
			\hline\hline
			0&0&0  \\ 
			\hline
			0&1&1  \\ 
			\hline
			1&0&1  \\
			\hline
			1&1&0 \\
			\hline
		\end{tabular}
	\end{center}
\end{frame}

\begin{frame}
	\frametitle{Sign flip}
	We apply $U_f$ (\textit{Quantum Oracle}) to the previous state $\psi^{[1]}$
	\begin{eqnarray}
		\psi^{[2]}&=& U_{f}\psi^{[1]}\nonumber\\
			&=&U_{f}\left(\sum_{j\in\{0,1\}^{n}}\alpha_{j}\left.|j\right\rangle _{n}\otimes\frac{1}{\sqrt{2}}\left(\left.|0\right\rangle -\left.|1\right\rangle \right)\right)\nonumber\\
		&=&U_{f}\left(\alpha_{l}\left.|l\right\rangle _{n}\otimes\frac{1}{\sqrt{2}}\left(\left.|0\right\rangle -\left.|1\right\rangle \right)+\sum_{
			j\in\{0,1\}^{n};\;
			j\neq l
			}\alpha_{j}\left.|j\right\rangle _{n}\otimes\frac{1}{\sqrt{2}}\left(\left.|0\right\rangle -\left.|1\right\rangle \right)\right)\nonumber\\
		&=&\left(\tikz[baseline]{
			\node[fill=blue!20,anchor=base, fill on=<2->,draw=red,rounded corners,draw on =<3>] (A)
			{$ -\alpha_{l}\left.|l\right\rangle _{n}$};
		} 
				 +\sum_{j\in\{0,1\}^{n};\;
				 	j\neq l}\alpha_{j}\left.|j\right\rangle _{n}\right)\otimes\tikz[baseline]{
			\node[fill=blue!20,anchor=base, fill on=<3>,draw=red,draw on =<3>,rounded corners] (AA)
			{$ \frac{1}{\sqrt{2}}\left(\left.|0\right\rangle -\left.|1\right\rangle \right)$};
		} 
	\end{eqnarray}
	\begin{tikzpicture}[
	remember picture,
	overlay,
	expl/.style={draw=red, thick=2pt,fill=blue!20,rounded corners},
	arrow/.style={red!80!black,ultra thick,->,>=latex}
	]
	\onslide<2->{\node[expl, xshift = -3cm](B) at (current page.center) {\Large Amplitud cambiada de signo!};}
	\onslide<2->{\draw[arrow]
	(A.north) to[bend left] (B.south);}
	\onslide<3->{\node[expl, xshift = 3cm](BB) at (current page.center) {\Large Extra qubit!};}
	\onslide<3->{\draw[arrow]
		(AA.north) to[bend right] (BB.south);}
	\end{tikzpicture}
\end{frame}


\begin{frame}
	\frametitle{Sign flip\footnote{This slide shows the result of applying $U_f$ but not how is applied. This is because this step of the algorithm depends on the specific problem.}}
	\begin{exampleblock}{Quantum Oracle}
		\begin{eqnarray}		
		\psi^{[2]} &=& U_f\psi^{[1]} \nonumber\\
		&=&[\frac{1}{2\sqrt{2}}\left.|000\right\rangle +\frac{1}{2\sqrt{2}}\left.|001\right\rangle +\frac{1}{2\sqrt{2}}\left.|010\right\rangle -\frac{1}{2\sqrt{2}}\left.|011\right\rangle\nonumber\\ &+&\frac{1}{2\sqrt{2}}\left.|100\right\rangle +\frac{1}{2\sqrt{2}}\left.|101\right\rangle +\frac{1}{2\sqrt{2}}\left.|110\right\rangle +\frac{1}{2\sqrt{2}}\left.|111\right\rangle ]\nonumber\\
		&\otimes&\frac{1}{\sqrt{2}}\left(\left.|0\right\rangle -\left.|1\right\rangle \right)\nonumber\\
		&=&\sum_{j\in\{0,1\}^{n}}\frac{1}{\sqrt{2^{n}}}\left.|j\right\rangle _{n}\otimes\frac{1}{\sqrt{2}}\left(\left.|0\right\rangle -\left.|1\right\rangle \right)\nonumber
		\end{eqnarray}
	\end{exampleblock}
\end{frame}
