\begin{frame}
	\frametitle{Cambio de signo}
	\begin{block}{\textit{Quantum Oracle}}
		Se define el operador $U_f$ tal que 
		\[U_{f}:\left.|j\right\rangle _{n}\otimes\left.|y\right\rangle _{1}\rightarrow\left.|j\right\rangle _{n}\otimes\left.|y\oplus f\left(j\right)\right\rangle _{1}\],
		donde $\oplus$ es el operador suma en modulo 2, y $f\left(j\right)=\left\{ \begin{array}{cc}
		1  & j=l\\
		0  & j\neq l
		\end{array}\right\} $
	\end{block}
	
	
	\begin{center}
		\begin{tabular}{|c c |c|} 
			\hline
			A & B & XOR \\ 
			\hline\hline
			0&0&0  \\ 
			\hline
			0&1&1  \\ 
			\hline
			1&0&1  \\
			\hline
			1&1&0 \\
			\hline
		\end{tabular}
	\end{center}
\end{frame}

\begin{frame}
	%\frametitle{Cambio de signo}
	\begin{eqnarray}
		U_{f}\psi&=&U_{f}\left(\sum_{j\in\{0,1\}^{n}}\alpha_{j}\left.|j\right\rangle _{n}\otimes\frac{1}{\sqrt{2}}\left(\left.|0\right\rangle -\left.|1\right\rangle \right)\right)\nonumber\\
		&=&U_{f}\left(\alpha_{l}\left.|l\right\rangle _{n}\otimes\frac{1}{\sqrt{2}}\left(\left.|0\right\rangle -\left.|1\right\rangle \right)+\sum_{\begin{array}{c}
			j\in\{0,1\}^{n}\nonumber\\
			j\neq l
			\end{array}}\alpha_{j}\left.|j\right\rangle _{n}\otimes\frac{1}{\sqrt{2}}\left(\left.|0\right\rangle -\left.|1\right\rangle \right)\right)\\
		&=&\left(\tikz[baseline]{
			\node[fill=blue!20,anchor=base, fill on=<2>] (A)
			{$ -\alpha_{l}\left.|l\right\rangle _{n}$};
		} 
				 +\sum_{\begin{array}{c}
			j\in\{0,1\}^{n}\nonumber\\
			j\neq l
			\end{array}}\alpha_{j}\left.|j\right\rangle _{n}\right)\otimes\tikz[baseline]{
			\node[fill=blue!20,anchor=base, fill on=<3>] (AA)
			{$ -\frac{1}{\sqrt{2}}\left(\left.|0\right\rangle -\left.|1\right\rangle \right)$};
		} 
	\end{eqnarray}
	\begin{tikzpicture}[
	remember picture,
	overlay,
	expl/.style={draw=orange,fill=blue!20,rounded corners},
	arrow/.style={red!80!black,ultra thick,->,>=latex}
	]
	\onslide<2->{\node[expl, xshift = -3cm](B) at (current page.center) {\Large Amplitud cambiada de signo!};}
	\onslide<2->{\draw[arrow]
	(A.west) to[out=200,in=180] (B.west);}
	\onslide<3->{\node[expl, xshift = 3cm](BB) at (current page.center) {\Large Extra qubit intacto!};}
	\onslide<3->{\draw[arrow]
		(AA.east) to[out=-200,in=-180] (BB.east);}
	\end{tikzpicture}
\end{frame}
